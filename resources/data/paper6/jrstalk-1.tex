% This is a paper by J.Stefanowski et al submitted to JRS 2014 in Spain

% Version 1.0 started Thursday, April 10th by JS
% Final - April 25 JS
% Revised - May 3 after JRS Chair's asks
% Final corrections and updates - May 6 after MK and CC email

\documentclass{llncs}


\begin{document}

\title{The Impact of Local Data Characteristics\\ on Learning from Imbalanced Data}

\author{Jerzy Stefanowski}
\institute{ Institute of Computing Science, Pozna\'n University of
Technology, 60-965~Pozna\'n,~Poland}
%\email{jerzy.stefanowski@cs.put.poznan.pl}}

\maketitle

\begin{abstract}
Problems of learning classifiers from imbalanced data are discussed. First,
we look at different data difficulty factors corresponding to complex
distributions of the minority class and show that they could be approximated
by analysing the neighbourhood of the learning examples from the minority
class. We claim that the results of this analysis could be a basis for
developing new algorithms. In this paper we show such possibilities by
discussing modifications of informed pre-processing method LN--SMOTE as well
as by incorporating types of examples into rule induction algorithm BRACID.
\end{abstract}

\section{Introduction}

Many difficult learning problems, from a wide variety of domains, involve
learning from imbalanced data, where at least one of the target classes
contains a much smaller number of examples than the other classes. This
class is usually referred to as the {\em minority class}, while the
remaining classes are denoted as {\em majority ones}. For instance, in
medical problems the number of patients requiring special attention is much
smaller than the number of patients who do not need it. Similar situations
occur in such domains as: fraud detection, risk management, technical
diagnostics, image recognition, text categorization or information
filtering. In all those problems, the correct recognition of the minority
class is of key importance. However, class imbalance constitutes a great
difficulty for most learning algorithms and often resulting classifiers are
biased toward the majority classes and fail to recognize examples from the
minority class.


While the difficulty with learning classifiers from imbalanced data has been
known earlier from applications, this challenging problem has received a
growing research interest in the last decade and a number of specialized
methods have already been proposed, for their review see, e.g.,
\cite{chawla_2005,He_2009,He,Herrera}.

In general, they may be categorized into {\em data level} and {\em algorithm
level} ones. The first group includes classifier-independent methods that
are used in the pre-processing step to modify the balance between classes,
so that standard algorithms can be used to learn better classifiers. These
methods are usually based on either adding examples to the minority class
(called {\em over-sampling}) or removing examples from the majority class
({\em under-sampling}). The other main category of, so called, algorithmic
methods involves modifications of either: learning phase of the algorithm or
classification strategy, construction of specialized ensembles, or
adaptation of cost sensitive learning. For their reviews see
\cite{chawla_2005,He_2009,He,Weiss}.

Although several specialized methods already exist, the identification of
conditions for their efficient use  is still an open research problem. It is
also related to more fundamental issues of better understanding the nature
of the imbalance data  and key properties of its underlying distribution.

Following related works \cite{Japkowicz,JapkowiczJo,Sanchez,Lopez2013} and
earlier studies of J. Stefanowski and  K.Napierala, J.Blaszczynski or Sz.
Wilk \cite{DAWAK,RSCTC2010,Krysia2012,Krysia2013} we claim that the high
imbalance ratio between the minority and majority classes is not the only
and not even the main reason of these difficulties. Other, as we call them,
{\em data difficulty factors}, referring to characteristics of class
distributions, are also influential. They include: decomposition of the
minority class into many rare sub-concepts, the effect of too strong
overlapping between the classes \cite{Prati,Sanchez} or a presence of too
many minority examples inside the majority class region. When these data
difficulty factors occur \textit{together} with class imbalance, they may
seriously hinder the recognition of the minority class, see e.g. a study
\cite{Lopez2013}.

In our earlier paper \cite{Krysia2012} we propose to capture these data
difficulty factors by considering the local characteristics of learning
examples from the minority class. More precisely, it is achieved by
analyzing the class distribution of examples from different classes inside a
{\em local neighborhood} of the considered example. Finding how many
examples from opposite classes are the neighbours of this example, the
degree of its difficulty could be estimated.

%As a result sub-regions of safe and unsafe examples could be discovered.

We claim that the proper analyzing of this neighborhood of learning examples
from the minority class could be the basis for developing new specialized
algorithms for imbalanced data. In this paper we "implement" this postulate
by considering representatives of two main categories of methods specialized
for imbalanced data. Firstly, we will apply the analysis of neighbours into
the new generalization of the most popular informed pre-processing method
SMOTE. Secondly, we will show that data difficulty factors modeled by types
of minority examples could be used inside the rule candidate generation
phase of the rule induction algorithm BRACID. Finally, we will discuss other
possible options of using the local information, in particular for ensembles
and highlight other future research directions of studying imbalanced data.

\section{Local Characteristics of Data Difficulty Factors and Identification of Example Types}

\label{sec:exampletypes}

Although many authors have experimentally shown that standard classifiers
met difficulties while recognizing the minority class, it has also been
observed that in some problems characterized by strong imbalance between
classes  standard classifiers are  sufficiently accurate. Moreover, the
discussion of data difficulty in imbalanced data still goes on, for its
current review see,
e.g.,~\cite{Herrera,He,Lopez2014,Krysia2012,stefanowski_2013}.

Some researchers have already noticed, that the {\em global class imbalance
ratio} (i.e. the cardinality of the majority class referred to the total number
of  minority class examples) is not necessarily the only, or even the main,
problem causing the decrease of  classification performance and focusing
only on this ratio may be insufficient for improving classification
performance. Besides the imbalanced ratio other data difficulty factors may
cause a severe deterioration of classifiers.

The experimental studies by Japkowicz {\em et al} with many artificial data
sets have clearly demonstrated that the degradation of classification
performance is also linked to the decomposition of the minority class into
many sub-parts containing very few examples \cite{Japkowicz,JapkowiczJo}.
They have shown that the minority class does not form a homogeneous, compact
distribution of the target concept but it is scattered into many smaller
sub-clusters surrounded by majority examples. In other words, minority
examples form, so called, {\em small disjuncts}, which are harder to learn
and cause more classification errors than larger sub-concepts.

Other factors related to the class distribution are also linked to the
effect of too strong {\em overlapping} between minority and majority class.
Experiments with artificial data have shown that increasing overlapping has
been more influential than changing the class imbalance ratio
\cite{Prati,Sanchez}. Yet another data factor, which influences degradation
of classifiers performance on imbalanced data, is presence of noisy examples
\cite{Grecy}. Experiments presented in~\cite{RSCTC2010} have also shown that
single minority examples located inside the majority class regions cannot be
treated as noise since their proper treatment by informed pre-processing may
improve classifiers. Moreover, studies as~\cite{stefanowski_2013} emphasize
that several data factors usually occur together in real world imbalanced
data sets.

These studies stress the role of the {\em local characteristics} of the
class distribution. However it could be modeled in different ways. Here, we
follow earlier works \cite{KubMat,Laurik,RSCTC2010,Krysia2013,Lopez2014} and
link data difficulty factors to {\em different types of examples} forming
the minority class distribution. It leads us to a differentiation between
safe and unsafe examples. {\em Safe examples} are ones located in the
homogeneous regions populated by examples from one class only. Other
examples are {\em unsafe} and more difficult for learning. Unsafe examples
are categorized into {\em borderline}~(placed close to the decision boundary
between classes), {\em rare cases}~(isolated groups of few examples located
deeper inside the opposite class), or {\em outliers}.  As the minority class
can be highly under-represented in the data,  we claim that the rare
examples or outliers, could represent a very small but valid sub-concepts of
which no other representatives could be collected for training. Therefore,
they cannot be considered as noise examples which typically are then removed
or re-labeled. Moreover, earlier works with graphical visualizations of
real-world imbalanced data sets \cite{Krysia2012} have confirmed this
categorization of example types.

The next question is how to automatically {\em identify these types of
examples} in real world data sets (with unknown underlying class
distributions). We keep the hypotheses \cite{Krysia2012} on role of the
mutual positions of examples and the idea of assessing the type of example
by analyzing class labels of the other examples in its {\em local
neighbourhood}. This neighbourhood of the minority example could be modeled
in different ways. In further considerations we will use an analysis of the
class labels among {\em k-nearest neighbours} \cite{Krysia2012,Krysia2013}.
This approach requires choosing the value of $k$ and the {\em distance
function}. In our previous considerations we have followed results of
analyzing different distance metrics \cite{laurikkala2004} and chose  the
HVDM metric ({\em Heterogeneous Value Difference  Metric})~\cite{hvdm}. Its
main advantage for mixed attributes is that it aggregates normalized
distances for qualitative and quantitative attributes. In particular,
comparing to other metrics it provides more appropriate handling of
qualitative attributes as instead of simple value matching it calculates
attribute value conditional probabilities by using a Stanfil and Valtz value
difference metric \cite{hvdm}.  Then, due to complexity of the distribution
of the minority class,  $k$ should be rather a small value. Experiments from
\cite{Krysia2012,Krysia2013} over many UCI data sets have showed that $k$ =
5 or 7 have led to good results.

Depending on the number of examples from the majority class in the
neighbourhood of the  minority example, we can evaluate whether this example
could be safe or unsafe (difficult) to be learned. If all, or nearly all,
its neighbours belong the minority class, this example is treated as the
safe example, otherwise it is one of unsafe types. For instance, consider
$k=5$.   In this case the type of example $x$ is defined as: if 5 or 4 of
its neighbours belong to the same class as $x$, it is treated as a safe
example; if the numbers of neighbours from both classes are similar
(proportions 3:2 or 2:3) -- it is a borderline example; if it has only one
neighbour with the same label (1:4) it is a rare case; finally if all
neighbours come from the opposite class (0:5) -- it is an outlier. Although
this categorization is based on intuitive thresholding, its results are
consistent with a probabilistic analysis of the neighbourhood, modeled with
kernel functions, as shown in \cite{Krysia2013}.

Our experiments with UCI imbalanced data sets \cite{Krysia2012,Krysia2013}
have also demonstrated that most of these real-world data do not include
many safe minority  examples. They rather contain all types of examples, but
in different proportions. On the other hand, most of majority class examples
have been identified as safe ones. Depending on the dominating type of
identified minority examples, the considered datasets have been labeled as:
safe, border, rare or outlier. As a large number of borderline examples
often occurred in many data sets, some of these data sets could be assigned
both to border and more difficult categories.

Moreover, the study \cite{Krysia2012} has shown that the classifier
performance could be related to the category of data. First, for the safe
data nearly compared single classifiers (SVM, RBF, k-NN, decision trees or
rules) perform quite well with respect to sensitivity, F-measure or G-mean.
The larger differentiation occurs for more unsafe data set. For instance,
SVM and RBF work much better for safe category, while rare or outlier data
strongly deteriorate their classification performance. On the other hand,
unpruned decision trees and k-NN work quite well for these unsafe data sets.
The similar analysis has been  carried out for the most representative
pre-processing approaches, showing that the competence area of each method
depends on the data difficulty level, based on the types of minority class
examples. Again in the case of safe data there are no significant
differences between the compared methods - even random over-sampling works
quite accurate. However, for borderline data sets Nearest Cleaning Rules
(methods filtering difficult majority examples \cite{Laurik}) performs best.
On the other hand, SMOTE \cite{Smote} and SPIDER \cite{DAWAK}, which can add
new examples to the data, have proved to be more suitable for rare and
outlier data sets.

For more details on the competence of each studied single classifier and
pre-processing methods see \cite{Krysia2013}. The similar analysis for
different generalizations of bagging ensembles, included specialized
solutions for class imbalances, have been carried out in the recent paper
\cite{bla_ste_idk_2013}. Finally, we will repeat our hypothesis that the
appropriate treatment of these types of minority examples within new
proposals of either pre-processing or classifiers should lead to improving
classification performance. We will show it in the next sections.

\section{Modifications of Informed Pre-processing Methods}

The simplest data pre-processing techniques are random over-sampling, which
replicates examples from the minority class, and random under-sampling,
which randomly eliminates examples from the majority classes until a
required degree of balance between classes is reached. However, random
under-sampling may potentially remove some important examples and simple
over-sampling may also lead to overfitting. Therefore, focused (also called
{\em informed}) methods, which attempt to take into account internal
characteristics of regions around minority class examples, were introduced,
as e.g. SMOTE \cite{Smote}, one-side-sampling \cite{KubMat}, NCR
\cite{Laurik} or SPIDER\cite{DAWAK} .

The most popular among the informed methods is SMOTE, which considers each
example from the minority class and generates new synthetic examples along
the lines between the selected example and some of its randomly selected
$k$-nearest neighbors from the minority class.  More precisely, let the
training set $S$ contain examples from the minority class $P$ and other
classes $N$. For each example $p_i \in P$ find its $k$ nearest neighbours
$x$ from class $P$. Depending on the other parameter of this method -- the
amount of over-sampling -- a given number of examples from these $k$ nearest
neighbours is randomly selected. Synthetic minority class examples are
generated in the direction of each. For numerical attributes the new
synthetic example is constructed as follows: compute the difference between
attributes describing the example $p_i$ and $x$ -- one of the selected
$k$-nearest neighbours; multiply this feature vector difference by $\delta$
-- a random number between 0 and 1; and add it to the attribute vector $p_i$
creating a new vector $x_{new} = p_i + (x - p_i)\cdot \delta$. For
qualitative attributes create a new example with the most common feature
values among $k$ nearest neighbours.

Although experiments have confirmed its usefulness (see e.g.
\cite{Batista,chawla_2005}), some of the assumptions behind this technique
could be still questioned. Two main shortcomings of SMOTE are: (1) treating
all minority examples in the same way while they may not be equally
important for learning classifiers (2) the possible over-generalization over
the majority class regions as SMOTE blindly generalizes regions of the
minority class without checking positions of the nearest examples from the
majority classes. Some researchers solve these problem by integrating SMOTE
with additional filtering steps (see e.g. \cite{Batista,Ramentol}), while
others modify SMOTE's internal strategies for selecting positions of
synthetic examples.

In this paper we focus on the recent proposal called Local Neighbourhood
extension of SMOTE (briefly LN-SMOTE) which is inspired by the analyzing
local data characteristics and earlier modifications of SMOTE
\cite{safelevel}. In this method the presence of the majority examples is
taken into account before generating synthetic examples by calculating a
special coefficient called a {\em safe level}. It is defined as the number
of other minority class examples among its $k$ nearest neighbours. The
smaller its value, the more unsafe is this example. This level $sl(p)$ is
calculated for the  example $p$, which is a seed for oversampling and as
$sl(x)$ for its randomly selected neighbour $x$. Unlike the standard SMOTE
and its generalizations as \cite{safelevel,Borderline}, in LN-SMOTE the
closest neighbours are calculated including also majority class examples.
Having information about values of both safe levels $sl(p)$ and $sl(x)$, the
range of positioning the synthetic example is modified. Only for equal both
levels the examples will be generated along the whole line joining $x$ and
$p$ in the same way as in the original SMOTE. If one safe level is greater
than other the position new example will generated closer the safer example
(more closer, the larger difference between these levels). Furthermore, in
case of the neighbour from the majority class, the range of random
overlapping is additionally limited not to come to close to the majority
examples. Situations of outliers (safe level equal 0) are also distinguished
by not putting the new examples for such a neighbour. Finally before
starting over-sampling, all majority examples being outliers inside the
minority class are identified by analysis content of $k$ neighbourhood -
they are removed from the learning set as they usually disturb the minority
class distribution.

The LN-SMOTE was introduced and experimentally studied in
\cite{maciejewski_2011}. Its comparative study against basic SMOTE and two
other related generalizations called Borderline-SMOTE \cite{Borderline} and
SL-SMOTE \cite{safelevel} applied with 3 different classifiers (J4.8, Naive
Bayes and k-NN) showed that it was the best pre-processing method. For
instance, Table \ref{tab:j48-fmeasure} summarizes results of F-measure for
these pre-processing methods applied to J4.8 decision tree, where Bord-SMOTE
denotes Borderline SMOTE and SL-SM denotes Safe Level SMOTE.

\begin{table}[!ht]
  \centering
  \caption{F-measure for the minority class for all compared methods used together with J48 classifier [\%]}
  \begin{tabular}{lccccc}
  \hline
  Data & None & SMOTE & Bord-SMOTE & SL-SM & LN-SMOTE \\
  \hline
balance-scale & 0.00 & 1.06 & 2.09 & 2.95 & 6.21 \\
car & 80.61 & 88.39 &
72.91 & 90.39 & 88.58 \\
cleveland & 19.29 & 21.35 & 22.89 & 21.33 & 26.70\\
cmc & 40.81 & 41.46 & 42.05 & 41.66 & 44.75\\
ecoli & 58.86 & 63.21 & 64.53
& 63.51 & 66.96 \\
 haberman & 30.36 & 38.41 & 41.50 & 37.33 & 42.20\\
 hepatitis & 49.20 & 49.86 & 51.23 &  52.57 & 54.42\\
 postoperative & 5.84 & 11.90 & 15,06 & 12.08 & 16.18\\
solar-flare & 28.79 & 27.13 & 28.32 & 29.62 & 31.6\\
transfusion & 47.27 &
48.07 & 49.79 & 49.22 & 50.30\\
yeast & 35.02 &  36.08 & 38.63 & 40.21 &
42.58\\ \hline
  \end{tabular}
  \label{tab:j48-fmeasure}
\end{table}


\section{Incorporating Types of Examples in Rule Induction}

Let us now consider using local characteristics of learning examples within
approaches modifying the algorithms. Decision rules, being the most human
readable knowledge representation, are particularly sensitive to class
imbalance, see e.g. conclusions from \cite{grzym-multi,KES,explore}.
Following some earlier methodological discussions \cite{Bracid,Weiss} the
standard algorithms for learning rule based classifiers share a number of
their principles which are useful for classification with respect to the
total accuracy but are limitations in case of class imbalances.

First, most algorithms induce rules using the top-down technique with
\textit{maximum generality bias}, which favors general rules however also
hinders finding rules for smaller sets of learning examples, especially in
the minority class. It is also connected with using \textit{improper
evaluation measures to guide the search} for best conditions. Typical
measures, as presented, e.g.,  in \cite{ksiazka,Sikora}, try to find a
compromise between the accuracy and generality of the rule which achieve
better values mainly for the majority class examples. Similar measures are
also often used to {\em prune} induced rules, which are particularly
inappropriate for small disjuncts or rare cases in the minority class where
rules may be constructed as the conjunction of many elementary conditions.
Thus, pruning of rules is guided mostly by measures referring more to the
majority class examples, neglecting the minority class specific
distributions \cite{An}.

Second, most algorithms use a \textit{greedy sequential covering} approach
\cite{Fruntz}, in which learning examples covered by the induced rule are
removed from the current set of considered examples. This approach may
increase the data fragmentation for the minority class and leads to the
induction of {\em weaker} rules, i.e. supported by a smaller number of
learning examples. The "weakness" of the minority rules could be also
associated with a third factor: {\em classification strategies}, where
minority rules have a smaller chance to contribute to the final
classification decision, see discussions in
\cite{grzym-multi,explore,Herrera}.

The above limitations concerns difficulties inside the algorithm. Recall
that one should also consider data difficulty factors referring to complex
distributions of the minority class, as presented in section 2.  Some
researchers have already proposed the extensions of rule based approaches
dedicated for class imbalance - for a comprehensive review see
\cite{Bracid}. However, most of these proposals addresses only a single or
at most a few of algorithmic or data factors.

Following these critical motivations  K.Napierala and J.Stefanowski have
introduced a new rule induction algorithm called BRACID (Bottom-up induction
of Rules And Cases for Imbalanced Data) which is specialized for the
classification of imbalanced data \cite{Bracid}. While constructing it we
have addressed several limitations mentioned above.

First we have decided to induce rules by {\em bottom-up generalization} of
the most specific rules representing single examples. Bearing in mind that
local algorithms could better learn the difficult decision boundaries (which
usually describe minority examples), this algorithm is a {\em hybrid} of
rule-based and instance-based {\em knowledge representations}. Due to this
representation it should also better deal with rare sub-concepts of the
minority class. Overcoming the problem of data fragmentation is also
connected with resigning from a greedy, sequential covering and top-down
induction technique. Inside the crucial operation of the bottom-up
generalization of the current rules the specific looking for the nearest
example has been applied.  The candidates for rules are temporarily added to
the current classifier and evaluated with the F-measure in a leaving-one-out
procedure. Such an approach allows us to evaluate and accept rules in a more
appropriate way for recognizing the minority class.  The final classifier
uses a {\em nearest rule strategy} to classify new coming examples
\cite{stefanowski02}, which has proved to be more appropriate for
recognizing minority classes than standard classification strategies too
much oriented to majority examples.

An important component of BRACID is also using information about the nature
of the neighbouring examples. Following the method presented in section 2,
we identify types of learning examples in each class. Unsafe outlier
examples from the majority classes are removed from the learning set as they
may hinder fragmentation of the minority class and then the induction of
more general minority rules. In case of an analogous situation for the
minority class, this example is not removed but it is checked as a candidate
for a rule generalization. Moreover for outliers, rare cases and borderline
minority examples we allow to analyse $k$ possible generalizations and to
choose best ones according to the F-measure evaluation. It allows us to
create more rules for the minority class in unsafe regions, as overlapping.
It should diminish the possibility of overwhelming the minority class with
the majority class rules in these difficult sub-regions.

\begin{table}[!ht]
  \centering
  \caption{G-mean for BRACID and other rule induction algorithms [\%]}
  \begin{tabular}{lcccc}
  \hline
  Data & BRACID & CN2 & RIPPER & MODLEM \\
  \hline
  abalone & 65.0 & 39.6 &  42.1 & 48.4 \\
  balance-scale & 56.7 & 2.9 & 1.9 & 0.0 \\
  cleveland & 57.4 & 0.0 & 25.8 & 19.2 \\
  cmc & 63.7 & 25.8 & 25.5 & 47.2 \\
  credit germ. & 61.1 & 55.3 & 43.8 & 56.3 \\
  ecoli & 83.1 & 28.4 & 58.7 & 56.8 \\
  haberman & 57.6 & 34.5 & 35.6 & 40.1 \\
  ionosphere & 91.2 & 87.0 & 87.4 & 89.2 \\
  vehicle & 93.5 & 51.3 & 91.9 & 91.6 \\
  transfusion & 63.9 & 34.2 & 26.6 & 52.9 \\
  \hline
  \end{tabular}
  \label{tab:bracid}
\end{table}



The experimental studies with components of BRACID, in particular this way
of incorporating types of examples have showed that it has improved the
evaluation measures comparing to the plain option of treating all learning
examples in the same way. For instance, for G-mean average improvements are
3.5 \% \cite{Bracid}. Finally, BRACID has been compared to a number of
state-of-the-art rule based classifiers (PART, RIPPER,C45rules,CN2, MODLEM,
RISE), instance-based classifier (K-NN) and some approaches dedicated for
class imbalances in the comprehensive experimental study over 22 imbalanced
data sets. The results showed that BRACID can better recognize the minority
classes than other compared algorithm - which is reflected by measures as
F-measure and G-mean. The selected results of G-mean are presented in Table
\ref{tab:bracid}.


Furthermore, using the categorization of data sets obtained with analysis
their local characteristics presented in section 2, we have found out that
the best improvements of BRACID are observed for unsafe data sets containing
many borderline examples from the minority class.


\section{Other Perspectives of Applying Local Data Characteristics}

The main message of this paper is to promote incorporating the information
about the local neighbourhood of a chosen minority class example in the
process of constructing and analyzing methods for learning classifiers from
imbalanced data. Although we have shown two such possibilities, still other
new directions are worth to be studied.

For instance, the proposed method for analysing $k$-neighbourhood of
minority examples as well as practically all informed pre-processing methods
have been considered for data with not so high number of attributes.
However, some applications of data mining in bio-medicine, text or
multimedia processing involve highly dimensional data sets. The use of
typical dissimilarity measures and  $k$-nearest neighbor classification on
such data sets may suffer from the curse of dimensionality problem as it has
been recently showed by Tomasev's research on, so called, {\it
hubness}-aware shared neighbor distances for high-dimensional k-nearest
neighbor classification \cite{Tomasev}. Thus, studying generalizations of
the presented methods for high dimensional data is still an open research
challenge.


In case of pre-processing methods as SMOTE, there still remain some
interesting questions on creation strategies for synthetic data, amount of
new data to create, better identification of the most appropriate
sub-regions of the minority class where to add new examples, avoidance of
introducing noise instead of valuable instances of the under-represented
class, distinguishing between noise and valuable outliers, and checking
whether synthetic examples are equally important as the real ones by
employing new evaluation measures while constructing and evaluating
classifiers.

Another point of view on informed pre-processing methods and the role of
generating large amounts of synthetic data could also come from specific
applications. Such studies as \cite{RSCTC2010,DAWAK} show that the degree of
over-sampling is quite high for many data sets, even if it is tuned just to
obtain a balance of cardinality of minority and majority classes. However,
in medical problems physicians could be reluctant to analyse so a high
number of artificial patients in their data set (i.e. their class of
interest could include more non existing patient's descriptions than real
clinical cases). Moreover, clinical experts often prefer to induce symbolic
classifiers due to their potential interpretability. For instance, if rule
induction algorithms are applied to data transformed by SMOTE, many rules in
the final classifier could be supported just by synthetic learning examples.
In spite of better classification performance such rules can be rejected by
clinical experts due to their artificial supports and expert could still
prefer to analyse rule referring to real facts in the original data.

These limitations open other perspectives on new informed preprocessing
methods that do not introduce synthetic examples as well as on new
specialized rule induction algorithms. Within the first perspective one can
consider generalizing hybrid methods as SPIDER \cite{DAWAK}, which should
better identify sub-regions where unsafe minority class examples should be
amplified with different degrees as well as better filter the majority
examples which too strongly influence these sub-regions. The other
perspective is partly considered in such rule induction algorithms as
EXPLORE \cite{explore}, BRACID \cite{Bracid} or ABMODLEM \cite{ABMODLEM}
(which allows to directly incorporate expert explanations as to classifying
difficult examples into the rule induction). However more extensive research
and medical practical case studies are still needed.

Considering types of example could be also applied to new ensembles
specialized for class imbalance. Most of current proposals are
generalizations of known techniques as bagging, boosting or random forests;
see their review in \cite{Galar,Liu}. Their  experimental results show that
modifications of bagging often outperform boosting generalizations or more
complex ensembles. While analyzing existing extensions of bagging one can
also notice that most of them employ the simplest random re-sampling
technique, as under-sampling or over-sampling, and, what is even more
important, they just modify bootstraps to simply balance the cardinalities
of minority and majority class.  However, in all of these extensions (see,
e.g., the Roughly Balanced Bagging \cite{Hido}), all examples are treated as
equally important while sampling them into bootstrap samples. We think that
drawing of minority examples should not be done in a pure blind random way
but it could be partly directed depending on the difficulty type of example.

In \cite{bla_ste_idk_2013} we have already proposed to change probability of
drawing different types of examples depending on the class distribution in
the neighbourhood of the example. This has led us to  the new type of
bagging ensemble, called Nearest Neighborhood Bagging. The recent
experiments \cite{bla_ste_ecml_2013} show that this new ensemble is
significantly better than existing over-sampling bagging extensions and it
is competitive to Roughly Balanced Bagging, which according to experiments
\cite{Hido,Taghi} is the most accurate under-sampling extension of bagging.
Nevertheless, several issues on: how much bootstrap samples should be
modified, the influence of filtering majority class examples, diversity of
bootstrap samples and the constructed classifiers, new techniques of their
aggregation should be still studied.

\noindent The research on this paper is partially supported by NCN grant.


\begin{thebibliography}{10}

\bibitem{An} An, A.: Learning classification rules from data. Computers and Mathematics with Applications, vol. 45, 737--748 (2003).

\bibitem{Grecy} Anyfantis,  D., Karagiannopoulos, M., Kotsiantis, S., Pintelas, P.:
Robustness of learning techniques in handling class noise in imbalanced
datasets. In:  Proc. of the IFIP Int. Federation for Information Processing
AIAI 2007,  21--28 (2007).

\bibitem{Batista} Batista, G., Prati, R., Monard, M.: A~study of the behavior of several methods for balancing machine learning training data. ACM SIGKDD Explorations Newsletter, vol. 6(1), 20--29 (2004).

\bibitem{safelevel} Bunkhumpornpat, C., Sinapiromsaran, K., Lursinsap, C.: Safe-Level-SMOTE: Safe Level Synthetic Over-Sampling TEchnique for Handling the Class Imbalance Problem. In: Proc. PAKDD 2009, Springer LNAI 5476, 475-482 (2009).

\bibitem{bla_ste_idk_2013} B\l{}aszczy\'nski, J., Stefanowski, J., Idkowiak L.:
Extending bagging for imbalanced data. In: Proc. of the 8th CORES 2013,
Springer Series on Advances in Intelligent Systems and Computing 226,
269--278 (2013).

\bibitem{bla_ste_ecml_2013}  B\l{}aszczy\'nski, J., Stefanowski, J., Szajek
 M.: Local Neighbourhood in Generalizing Bagging for Imbalanced Data. In
 Proc. of COPEM 2013 - Solving Complex Machine Learning Problems
with Ensemble Methods Workshop at ECML PKDD 2013, Praque, 10--24 (2013).

\bibitem{chawla_2005} Chawla, N.: Data mining for imbalanced datasets: An overview. Chapter in Maimon O., Rokach L. (eds.): The Data Mining and Knowledge Discovery Handbook, Springer, 853--867
(2005).

\bibitem{Smote} Chawla, N., Bowyer, K., Hall, L., Kegelmeyer, W.: SMOTE: Synthetic Minority Over-sampling Technique. J. of Artificial Intelligence Research, 16, 341-378 (2002).

\bibitem{vdm} Cost, S., Salzberg, S.: A Weighted Nearest Neighbor Algorithm for Learning with Symbolic Features. Machine Learning Journal, vol. 10 (1),  1213--1228 (1993).

\bibitem{Herrera} Fernandez, A., Garcia, S., Herrera, F.: Addressing the Classification
with Imbalanced Data: Open Problems and New Challenges on Class
Distribution. In. Proc. HAIS Conf. (part. 1),  1-10, (2011).


\bibitem{Fruntz} Furnkranz, J.:  Separate-and-conquer rule learning. Artificial Intelligence Review, vol. 13(1), 3--54, (1999).

\bibitem{ksiazka} Furnkranz, J., Gamberger D., Lavrac N.: Foundations of Rule Learning. Springer, (2012).

\bibitem{Galar} Galar, M.,  Fernandez, A.,  Barrenechea, E.,  Bustince, H.  Herrera, F.: A Review on Ensembles for the Class Imbalance Problem: Bagging-, Boosting-, and Hybrid-Based Approaches. IEEE Transactions on  Systems, Man, and Cybernetics, Part C: Applications and Reviews, vol. 99, 1--22 (2011).

%\bibitem{Herrera} Garcia, S., Fernandez, A., Herrera, F.: Enhancing the effectiveness and interpretability of decision tree and rule induction classifiers with evolutionary training set selection over imbalanced problems. Appl. Soft Computing 9 (4), 1304-1314, (2009).

\bibitem{Sanchez} Garcia, V., Sanchez, J.S., Mollineda, R.A.: An empirical study of the behaviour of classifiers on imbalanced and overlapped data sets. In: Proc. of  Progress in Pattern Recognition, Image Analysis and Applications 2007, Springer, LNCS, vol. 4756,  397--406 (2007).

\bibitem{grzym-multi} Grzymala-Busse, J.W., Goodwin, L.K., Grzymala-Busse, W., Zheng, X.: An approach to imbalanced data sets based on changing rule strength. Proceedings of Learning from Imbalanced Data Sets, AAAI Workshop   at the 17th Conference on AI, 69--74, (2000).

\bibitem{KES} Grzymala-Busse, J.W., Stefanowski, J., Wilk, S.: A~comparison of two approaches to data mining from imbalanced data. Journal of Intelligent Manufacturing, vol. 16 (6), 565-574 (2005).

\bibitem{Borderline} Han H., Wang W., Mao B.: Borderline-SMOTE: A New Over-Sampling Method in Imbalanced Data Sets Learning. In Proc. ICIC, Springer LNCS vol. 3644, 878-887 (2005).

\bibitem{He_2009} He, H., Garcia, E.: Learning from imbalanced data. IEEE Transactions on Data and Knowledge Engineering, vol. 21 (9), 1263--1284 (2009).

\bibitem{He} He, H., Yungian, Ma (eds): Imbalanced Learning. Foundations, Algorithms and Applications. IEEE - Wiley, (2013).

\bibitem{Hido} Hido, S., Kashima, H.: Roughly balanced bagging for imbalance data. Statistical Analysis and Data Mining, vol. 2 (5-6), 412--426 (2009).

\bibitem{Japkowicz} Japkowicz, N.: Class imbalance: Are we focusing on the right issue? In: Proc. II Workshop on Learning from Imbalanced Data Sets, ICML Conf., 17--23, (2003).

\bibitem{JapkowiczJo} Jo, T., Japkowicz, N.: Class Imbalances versus small disjuncts. ACM SIGKDD Explorations Newsletter, 6 (1),  40--49 (2004).

\bibitem{Taghi} Khoshgoftaar, T., Van Hulse, J., Napolitano, A.:
Comparing boosting and bagging techniques with noisy and imbalanced data.
IEEE Transactions on Systems, Man, and Cybernetics--Part A, vol. 41 (3),
552--568 (2011).

\bibitem{KubMat} Kubat, M., Matwin, S.: Addresing the curse of imbalanced training sets: one-side selection. In:  Proc. of the 14th Int. Conf. on Machine Learning ICML-97, 179-186 (1997).

\bibitem{Laurik} Laurikkala, J.: Improving identification of difficult small classes by balancing class distribution. Tech. Report A-2001-2, University of Tampere, (2001).

%\bibitem{Lewis94} Lewis, D., Catlett, J.: Heterogenous uncertainty sampling for
%supervised learning. In: Proc. of 11th Int. Conf. on Machine Learning,
%148--156 (1994).

\bibitem{Liu} Liu A., Zhu Zh: Ensemble methods for class imbalance learning. In He, H., Yungian Ma. (eds): Imbalanced Learning. Foundations, Algorithms and Apllications. Wiley, 61-82 (2013).

\bibitem{laurikkala2004} Lumijarvi, J., Laurikkala, J., Juhola, M.: A comparison of different heterogeneous proximity functions and Euclidean distance. Stud Health Technol. Inform., 107 (Pt 2), 1362--1366 (2004).

\bibitem{Lopez2013}Lopez, V., Fernandez, A.,  Garcia, S., Palade, V., Herrera, F.: An Insight into Classification with Imbalanced Data: Empirical Results and Current Trends on Using Data Intrinsic Characteristics. Information Sciences vol. 257, 113-141,(2014).

\bibitem{Lopez2014} Lopez, V., Triguero, I.,  Garcia, S.,Carmona, C., Herrera, F.:
Addressing imbalanced classification with instance generation techniques:
IPADE-ID. Neurocomputing vol. 126, 15-28 (2014).

\bibitem{maciejewski_2011} Maciejewski, T., Stefanowski, J.:
Local neighbourhood extension of SMOTE for mining imbalanced data. In: Proc.
IEEE Symp. on Computational Intelligence and Data Mining,  104--111 (2011).

\bibitem{Krysia2013} Napierala, K.: Improving rule classifiers for imbalanced data. Ph.D. Thesis. Poznan University of Technology, (2013).

\bibitem{RSCTC2010} Napierala, K., Stefanowski, J., Wilk, Sz.: Learning from Imbalanced Data in Presence of Noisy and Borderline
Examples.
In Proc. of 7th Int. Conf. RSCTC 2010, Springer, LNAI vol. 6086, 158-167 (2010).

\bibitem{ABMODLEM} Napierala, K., Stefanowski, J.: Argument Based Generalization of MODLEM Rule Induction
Algorithm. In Proc. of 7th Int. Conf. RSCTC 2010, Springer, LNAI vol. 6086,
138-147 (2010).

\bibitem{Krysia2012} Napierala, K., Stefanowski, J.: The influence of minority class distribution on learning from imbalance data. In. Proc. 7th Conf. HAIS 2012,  LNAI vol. 7209, Springer, 139-150 (2012).

\bibitem{Bracid}  Napierala, K., Stefanowski, J.: BRACID: a comprehensive approach to learning rules from imbalanced data.
Journal of Intelligent Information Systems, vol. 39 (2), 335-373 (2012).

\bibitem{Prati}  Prati, R., Batista, G., Monard, M.: Class imbalance versus class overlapping: an analysis of a learning system behavior. In Proc. 3rd Mexican Int. Conf. on Artificial Intelligence, 312--321 (2004).

\bibitem{Ramentol}  Ramentol, E., Caballero, Y., Rafael Bello, Herrera, F.:
SMOTE-RSB *: a hybrid preprocessing approach based on oversampling and
undersampling for high imbalanced data-sets using SMOTE and rough sets
theory. Knowledge Inform. Systems 33(2), 245-265 (2012).

\bibitem{Sikora} Sikora, M., Wrobel, L.: Data-driven adaptive selection of rule quality measures for improving rule induction and filtration algorithms. Int. J. General Systems vol. 42(6),  594-613 (2013)

\bibitem{stefanowski02} Stefanowski, J.: On combined classifiers, rule induction and rough sets. Transactions on Rough Sets, 6, 329--350 (2007).

\bibitem{stefanowski_2013} Stefanowski, J.: Overlapping, rare examples and class decomposition in learning classifiers from imbalanced data. In S.Ramanna, L.C. Jain, and R.J. Howlett (eds), Emerging Paradigms in Machine Learning, 277-306 (2013).

\bibitem{DAWAK} Stefanowski, J., Wilk, Sz.: Selective pre-processing of imbalanced data for improving classification performance. In: Proc. of the 10th Int. Conf. DaWaK 2008. LNCS, vol. 5182. Springer,  283--292 (2008).

\bibitem{explore} Stefanowski J., Wilk Sz.: Extending rule-based classifiers to improve recognition of imbalanced   classes. In:
Ras, Z., Dardzinska, A. (eds): Advances in Data Management,  Studies in
Computational Intelligence, Springer, vol. 223, 131--154 (2009).

\bibitem{Tomasev} Tomasev, N., Mladenic, D.: Class imbalance and the curse of minority hubs -
Knowledge-Based Systems, vol. 53, 157�-172, (2013).

\bibitem{Weiss}  Weiss, G.M.: Mining with rarity: a unifying framework. ACM SIGKDD Explorations Newsletter, vol. 6 (1), 7-19 (2004).

\bibitem{hvdm} Wilson, D.R., Martinez, T.R.:  Improved heterogeneous distance functions. Journal of Artifical Intelligence Research, vol. 6, 1-34 (1997).

\end{thebibliography}



\end{document}
