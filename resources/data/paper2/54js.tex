% Paper prepared for AI-METH 2002 Conference, Gliwice
% Paper Version - 1.0, last changes 27 August 2002 done by Jurek
% Paper Version - 1.1, last changes 10 September 2002 done by Jurek
% Paper Version - 2.0, last changes 11 September 2002 done by Jurek
% Paper Version - 2.1, last changes 12 September 2002 done by Jurek

\documentclass{AIMeth02}

\begin{document}

\title{Feature selection in the $n^2$-classifier applied for multiclass problems}

\author{
{\bf   %%   Put the name of the 1st author:
Jacek Jelonek and Jerzy Stefanowski
}\\    %%   the Institution:
{\it Institute of Computing Science, Pozna\'n University of Technology,}%
\\     %%   the adress:
{\it ul. Piotrowo 3A, 60-965 Pozna\'n,  Poland}
\\     %%   and e-mail:
{e-mails: Jacek.Jelonek@cs.put.poznan.pl,  Jerzy.Stefanowski@cs.put.poznan.pl}
%
}

\abstr{The paper discusses solving multiclass learning problems by
a multiple classification system, called the $n^2$-classifier. Its
architecture and work is based on the set of binary classifiers -
one for each pair of decision classes. This approach is extended
by introducing feature selection for each base classifier. We
consider two different approaches to conduct a search for the most
relevant features. They are evaluated in an experimental study.
\keywords{ machine learning, multiple classifiers, feature
selection, genetic algorithms}}


\maketitle

\section{Introduction}
\label{Introduction}

Creating systems that automatically learn from provided examples
is one of the main problems considered in machine learning and
artificial intelligence. In particular, one of common tasks is
\emph{supervised learning}, where learning process aims at finding
function that assigns learning examples, each described by a fixed
set of attributes (features), to known a priori decision classes.
Such a function expresses knowledge obtained by an algorithm from
examples and it can be successively used to classify new,
previously unobserved, instances. In this sense learning process
results in creating \emph{classification system} -- shortly called
\emph{classifier} \cite{Weiss}. Typical measure used to evaluate
such systems is \emph{classification accuracy}, i.e. percentage of
correctly classified testing examples.

Recently, there has been observed a growing interest in increasing
classification accuracy by integrating different classifiers
into one composed classification system In proper circumstances
such composed system should better classify new (or testing)
examples than its component classifiers used independently.
The integration of \emph{multiple classifiers} has been approached
in many ways, for some review see, e.g.
\cite{Dietrich,Gama,Jelonek00,Stef01}. Experimental evaluations
have confirmed that the use of multiple classifiers leads to
improving classification accuracy in many problems.

In this paper, we focus our attention on specific multiple
classifiers used to solve \emph{multiclass learning problems}.
This problem involves finding a classification system that assigns
examples into  $n$ decision classes, where $n>2$. Although the
standard way to solve it includes the direct use of the multiclass
learning algorithm such as, e.g. algorithm for inducing decision
trees, neural networks, or instance-based algorithms, there exist
however more specialized methods dedicated to this problem. Such
approaches as, e.g. one-per-class method,
%distributed output codes classification schemes,
error-correcting techniques (ECOC), or
pairwise coupling can outperform the direct use of the single
multiclass learning algorithms  \cite{DieB,Friedman}.

Within the framework of multiple classifiers dedicated to solve multiclass
learning problems, we introduced the new system, called $n^2$-\emph{classifier}
\cite{JelStef97}.
It is composed of $(n^2-n$)/2 \emph{binary base classifiers}. Each base
classifier is specialized to discriminate respective pair of decision classes only.
In the learning phase each base classifier is constructed by the same learning algorithm
but on the subset of learning examples belonging to the given pair of classes. In the
classification phase, a new example is classified by applying  its description to all
base classifiers. Then, their predictions are aggregated to a final classification
decision using a majority voting rule.

In our former research, we were interested in the problem of
choosing proper learning algorithm to create base classifiers in
the $n^2$-classifier. In \cite{JelStef97,JelStef98,Stef01hab} we
performed series of computation experiments where the influence of
the choice learning algorithm on classification performance was
examined. Four different algorithms were used to learn:  decision
trees, decision rules (using MODLEM algorithm \cite{Stef01hab}),
neural networks
%(typical feed forward multi-layer network learned with backpropagation),
and instance based learning (based on $k$ nearest neighbor
principle -- abbreviated as $k\!-\!NN$). The obtained results
showed that the classification accuracy of the $n^2$-classifier is
significantly better than the accuracy of a respective single
multiclass classifier (obtained by the algorithm of the same type
as used to create base classifiers) for three algorithms: decision
trees, rules and neural networks. On the contrary, using $k\!-\!N$
algorithm did not resulted in so encouraging improvement. We have
suspected that its the worst performance could result from the
fact that this learning algorithm  treats all features
(attributes) as equally important while other approaches (in
particular decision trees and rules) have  inherent capability of
reducing the \emph{irrelevant features} what may help with
defining proper subspaces of features for efficient solving
two-class problems.

In this paper,
%we want to experimentally verify this hypothesis
%about possible increasing the accuracy of the $n^2$-classifier
%when dedicated subsets of features are used for discriminating
%each pair of decision classes. The
the $n^2$-classifier is extended by connecting learning phase of
its base $k\!-\!NN$ classifiers with methods of \emph{feature
selection}. We consider two different approaches. % to the feature selection.
In the first approach, the best feature subsets are looked for
each pair of classes independently of other pairs. Moreover, these
subsets are evaluated locally with the respect to their usefulness
for correct discriminating examples from the given pair of
classes. In the other approach the feature subsets  are searched
simultaneously for all pairs of classes and they are evaluated by
the global classification accuracy. In the experiments, both these
approaches are compared on several benchmark data sets against the
single multiclass classifier and the $n^2$-classifier working with
the complete set of features.


\section{The $n^2$-classifier}

\subsection{An architecture and learning of the $n^{2}$-classifier}

The $n^2$-classifier is composed of $(n^2-n)/2$ base binary classifiers.
The main idea is to discriminate each pair of the classes: ($i,j
$), $i,j \in [ 1..n ], i \neq j$ , by an independent binary classifier $
C_{ij}$. Each base binary classifier $C_{ij}$ corresponds to a pair of two
classes $i$ and $j$ only. Therefore, the specificity of the training of each
base classifier $C_{ij}$ consists in presenting to it a subset of the entire
learning set that contains only examples coming from classes $i$ and $j$. The
classifier $C_{ij}$ yields a binary classification indicating whether a new
example, {\bf x}, belongs to class $i$ or to class $j$. Let us denote by $
C_{ij}$({\bf x}) the classification of an example {\bf x} by the base
classifier $C_{ij}$. In following description of the $n^2$-classifier we
assume that $C_{ij}$({\bf x}) = 1 means that example {\bf x} is classified
by $C_{ij}$ to class $i$, otherwise ($C_{ij}$({\bf x})= 0) {\bf x} is
classified to class $j$.

The complementary classifiers: $C_{ij}$ and $C_{ji}$ (where $i,j
\in \;  <1 \ldots n>; \; i \neq j$) solve the same classification
problem -- they are trained on the same set of examples in order
to discriminate between class $i$-th and $j$-th. So, they are
equivalent ($C_{ij} \equiv C_{ji}$) and inside the architecture of
the $n^2$-classifier it is sufficient to use only ($n^2$ - $n$)/2
classifiers $C_{ij} (i < j)$, which correspond to all combination
of pairs of $n$ classes.


An algorithm providing final classification assumes that a new
example {\bf x} is applied to all base classifiers $C_{ij}$. As a
result, their binary predictions $C_{ij}$({\bf x}) are computed.
The final classification should be obtained by a proper
aggregation of these predictions. The simplest aggregation rule is
based on finding a class that wins the most pairwise comparisons.
The more sophisticated approach may use a \emph{weighted} majority
voting rules, where the vote of each classifier is modified by its
credibility. It may be calculated as its classification accuracy
during learning phase, see e.g. \cite{JelStef98}.



The quite similar approach was independently introduced by
Friedman \cite {Friedman}. Then it was extended and experimentally
studied in \cite{Hastie}. Hastie and Tibshirani called the
extended  classification model as {\em classification by pairwise
coupling}. As it has been indicated in experiments
\cite{Friedman,Hastie} such an integration of binary classifiers
performs usually better than the single classification model.


\subsection{Feature selection inside $n^{2}$-classifier}

The $n^2$-classifier may be more accurate than the standard
multiclass single algorithms for such multiclass problems, where
decision concepts are described by ''complex'' target functions
which could be difficult to learn. In this case, each of pairwise
decisions is likely to be (much) simpler function of input
attributes (features) and can be easier approximated. We  suspect
that the success of learning of simpler and more accurate pairwise
decision boundaries between each pair of classes is also connected
with using the limited number of the most \emph{relevant features}
describing examples. The presence of too many irrelevant features
may decrease the classification performance. Therefore some
algorithms with capability of reducing the influence of irrelevant
features (e.g. decision trees) could be more appropriate to use in
the framework of the $n^2$-classifier than algorithms in which all
features are treated as equally important. Indeed, the {\em k-NN}
algorithm is known to be sensitive to existence of irrelevant
features.

Our previous experimental studies
\cite{JelStef97,JelStef98,Stef01hab} showed that the
classification performance of the $n^{2}$-classifiers based on
decision trees, rules and neural networks was significantly better
than the single classifier approach, while the use of simple
instance-based learning algorithm was not so encouraging.
Therefore,  we decide to extend the $n^2$-classifier based on
$k\!-\!NN$ by connecting learning phase of its base classifiers
with techniques of feature selection.

Let us notice that problem of feature selection also occurs for
many standard learning algorithms, which may perform poorly when
faced with too many irrelevant feature. Several methods have
already been proposed, see e.g. review in
\cite{Dash,kohavi94,Kohavi95}. In our classifier we employ, the so
called, \emph{wrapper model} \cite{kohavi94}, as a general
framework. In this model, the given feature subset selection
algorithm conducts a search for the good subset using classifier
itself as a part of the \emph{evaluation function}. It means that
the subset of features is provided to train the classifier whose
classification accuracy is estimated -- usually by a
cross-validation technique. Notice however, that classification
accuracy of the final composed $n^2$-classifier, containing chosen
sets of features, should also be evaluated on the verification
examples, which were not used in the learning phase to select
particular features \cite{Kohavi95}.

We consider two different approaches to conduct a search for the
best  subsets of features within above framework.

The first approach is "\emph{local}" in this sense that the
feature subsets will be looked for each pair of classes
independently of other pairs and these subsets will be evaluated
by means of the classification accuracy calculated for examples
belonging to this pair of classes only. The search algorithm,
which looks through the space of feature subset for the given base
classifier, will be a \emph{forward stepwise selection}. It starts
search with an empty set of features and successively adds the one
with the best classification performance. The process of adding
features is stopped when no improvement of the evaluation function
is observed. As it is a version of this greedy choice, we extend
this technique by introducing a {\em beam search} strategy. It
maintains a fixed-size collection of the best performing subsets
of features. This collection is updated in each iteration of the
forward selection, for more details see \cite{JelStefAI}. This
algorithm will be shortly called \emph{FBFS}.

On the contrary to the previous approach ("local" one), in which
features of all base classifiers are searched independently, in
this "\emph{global}" approach feature search is conducted for all
base classifiers at the same time. Moreover, the feature subsets
are evaluated by means of the classification accuracy calculated
for the $n^2$-classifier and examples belonging to all classes. As
even for small instances of feature search problems the number of
possible solutions is huge, meta heuristic approach should be
applied. In our case we decided to use a \emph{genetic algorithm}.

Let us remind that a genetic algorithm is an iterative procedure
that consists of a constant-size population of individuals, each
one represented by a finite string of symbols, known as the
\emph{genome}, encoding a possible solution in a given problem
space \cite{Goldberg}. The standard genetic algorithm proceeds as
follows: an initial population of individuals is generated at
random or heuristically. Every evolutionary step, known as a
generation, the individuals in the current population are decoded
and evaluated according to some predefined quality criterion,
referred to as the \emph{fitness}, or fitness function. To form a
new population (the next generation), individuals are selected
according to their fitness. Usually individuals are selected with
a probability proportional to their relative fitness. This ensures
that the expected number of times an individual is chosen is
approximately proportional to its relative performance in the
population. Selection alone cannot introduce any new individuals
into the population, i.e., it cannot find new points in the search
space. These are generated by genetically-inspired operators, of
which the most well known are \emph{crossover} and
\emph{mutation}. Crossover is performed with a given probability
between two selected individuals, called parents, by exchanging
parts of their genomes to form two new individuals, called
offspring. The mutation operator is introduced to prevent
premature convergence to local optima by randomly sampling new
points in the search space. It is carried out by flipping bits at
random. Genetic algorithms are stochastic iterative processes that
are not guaranteed to converge; the termination condition may be
specified as some fixed, maximal number of generations or as the
attainment of an acceptable fitness level.

In order to encode usage of the features for all base classifiers,
each genome consists of a set of binary strings. A number of
strings included in a genome equals to the number of the base
classifiers. So, each string refers to a particular base
classifier and indicates which features should be applied for it
('1' in the string means feature should be used and '0'
otherwise). Evaluation of the genome is done by estimating of a
classification accuracy of the $n^2$-classifier, which is created
taking into account the features defined by the genome. This
estimation is done on the basis of "inner" cross-validation, i.e.
it uses only samples from the learning set of the main
cross-validation loop. The crossover operator is applied for each
string of a parents� genome and exchange them after a randomly
selected crossover point. The mutation operator randomly flips one
bit in each string in the genome according to the given
probability. The genetic algorithm is terminated after predefined
number of generations.



\section{Experiments}

The aim of the experimental study is to check how much two
different techniques of feature selection, discussed in this
paper, can increase classification accuracy of the
$n^2$-classifier. The $k\!-\!NN$ algorithm was employed to create
classifiers in all experiments. We compare performance of:
\begin{enumerate}
\item The single classifier used for all features without any
selection.
\item The $n^2$-classifier, where base classifiers are learned with the complete
set of features.
\item The $n^2$-classifier with "local" feature selection
 performed by the FBFS algorithm.
\item The $n^2$-classifier with "global" feature selection
performed by the genetic algorithm.
\end{enumerate}

\noindent
 All experiments have been performed on the following
 benchmark data sets (concerning
multiclass problems): \emph{Automobile, Ecoli, Glass, Meta-data,
Yeast} and \emph{Zoo}. Their characteristics is given in the Table
\ref{tab1}. They are coming form Machine Learning Repository at
the University of California at Irvine \cite{irvine}. Two original
data sets were slightly modified. For \emph{Meta} data set the
continuous decision attribute was discretized using the following
cut points: 6, 13, 20 and 50. In case of \emph{Automobile} data
set, examples containing missing values were removed.  The
classification accuracy was estimated by stratified version of
10-fold cross-validation technique, i.e. the training examples
were partitioned into 10 equal-sized blocks with similar class
distributions as in the original set. In case of employing the
wrapper model inside the n$^2$-classifier, the additional "inner"
cross validation was used on the learning sets to evaluate feature
subsets. Classification accuracies for all four compared
classifiers are presented in the Table \ref{accuracy}.



\begin{table}
\caption{Data sets used in the experiments}
\label{tab1}
\vspace{2pt}
\begin{tabular}{llll}
\hline
 Data set & Number of & Number of & Number of \\
 & examples & classes & attributes \\ \hline
 Automobile & 159 & 6 & 43 \\
Ecoli & 336 & 8 & 7 \\ Glass & 214 & 6 & 9 \\
 Meta-data & 528 & 5
& 43 \\ Yeast & 1484 & 10 & 8 \\
 Zoo &  101 & 7 & 16 \\
\hline
\end{tabular}
\end{table}



\begin{table}
\caption{Classification accuracy of compared $k\!-\!NN$ \newline
based classifiers (expressed in \%)} \vspace{2pt} \label{accuracy}
\begin{tabular}{lllll}
\hline
Data & Single & $n^2$ all & $n^2$ with & $n^2$ with \\
        & classifier &  features &  FBFS & gen. alg. \\
\hline
 Automobile & 77.9 & 76.7 &
85.9 & 86.1 \\
 Ecoli & 81.0 & 81.3 &
82.0 & 81.5 \\ Glass & 68.8  & 68.5  & 71.0  &  70.5\\ Meta & 40.6
& 42.1 & 48.0 & 47.8 \\ Yeast & 52.7 & 53.3 & 52.9 & 54.0
\\ Zoo & 96.9 & 96.9 & 93.9 & 97.0 \\ \hline
\end{tabular}
\end{table}



\section{Conclusions}


Solving multiclass learning problems by the $n^2$-classifier is
discussed. Its architecture and work is based on ($n^2 -n$)/2
binary homogeneous classifiers -- one for each pair of classes. In
this paper the $n^2$-classifier has been extended by introducing
feature selection for each pair of classes. Two different
approaches for selecting features, "local" and "global", have been
introduced in the binary classifiers trained by instance-based
learning algorithm ($k\!-\!NN$). The results of their experimental
evaluation are discussed in the following points:
\begin{enumerate}
\item In general, the classification performance of the $n^2$
classifier based on $k\!-\!NN$ (with all features) and the single
classifier is comparable.
\item The use of features selection performed by the $FBFS$
algorithm improves the classification accuracy, comparing to the
single multiclass classifier, on 4 of 6 data sets; The highest
improvements are for \emph{Automobile} and \emph{Meta data}.
\item The use of features selection performed by the genetic
algorithm improves the classification accuracy, comparing to the
single multiclass classifier, on 4 of 6 data sets; The highest
improvements are also observed  for \emph{Automobile} and
\emph{Meta data}.
\item Comparing FBFS and genetic versions of the feature selection we
observe the similar classification results.
\end{enumerate}

\noindent Let us comment that the improvement of classification
accuracy for the $k\!-\!NN$ based $n^2$-classifier was obtained in
case of adding feature selection technique only. It confirms our
hypothesis that looking for dedicated subsets of features, which
discriminate pairs of decision classes, is crucial in the
$n^2$-classifier. Let us remind that in our previous experiments
such learning algorithms, which have inherent capability of
reducing irrelevant features (e.g. decision trees or rules), also
led to an improvement of the classification accuracy for the
$n^2$-classifier \cite{JelStef98}. These results could open a new
interesting research problem concerning \emph{constructive
induction} in the $n^2$-classifier, i.e. transformation of
original features into new more discriminative ones.


Another interesting research issue concerns rather small
difference between the use of both feature selection techniques.
The intuition could indicate that the "global" approach, focused
on more global evaluation function, should lead to much better
results than the "local" one. However, the "local" approach is
accurate at the similar level (except data set \emph{Zoo}) but it
is much simpler and less demanding from the computation points of
view.

To conclude, the use of the$n^2$-classifier leads to increasing
classification accuracy for multiclass learning problems. However,
the structure of the multiple classifier seems to be more
complicated and difficult to interpret by human than the single
model. Let us also remind that computation costs of training the
$n^2$-classifier with feature selection is higher than the
standard approach. On the other hand, one can gain increased
accuracy, which would be worth these additional costs for some
problems.

\vspace{6pt}

\noindent {\bf Acknowledgment}: The authors want to acknowledge
support from State Committee for Scientific Research, research
grant no. 8T11F 006 19. The software for this work used the GAlib
genetic algorithm package written by Matthew Wall at MIT.

\begin{thebibliography}{17}

\bibitem{irvine}  Blake C., Koegh E., Mertz C.J.,
Repository of Machine Learning, University of California at Irvine
1999 [URL: http://www.ics.uci.edu/~mlearn/MLRepositoru.html].

\bibitem{Dash}
Dasf M., Liu H., Feature selection for classification. {\em
Intelligent Data Analysis}, {\bf 1}(3) (1997), 131--156.

\bibitem{Dietrich} Dietrich T.G.,  Ensemble methods in machine learning.
{\em Proc. of 1st Int. Workshop on Multiple Classifier Systems}
(2000), 1--15 .

\bibitem{DieB}  Dietterich T.G., Bakiri G., Solving muliclass learning
problems via error- correcting output codes. {\em Journal of
Artificial Intelligence Research}, {\bf 2} (1995), 263--286.

\bibitem{Friedman}  Friedman J., Another approach to polychotomous
classification, Technical Report, Stanford University, 1996.

\bibitem{Gama}  Gama J., Combining classification algorithms. Ph.D. Thesis,
University of Porto, 1999.

\bibitem{Goldberg}
Goldberg D.E., \emph{Genetic algorithms in search, optimization
and machine learning}, Addison-Wesely Publishing, Massachusetts,
1989.

\bibitem{Hastie}  Hastie T., Tibshirani R., Classification by pairwise
coupling. {\em Proceedings} NIPS97, 1997.

\bibitem{Jelonek00} Jelonek J., Zastosowanie zlozonego systemu klasyfikacyjnego
$n^2$ z mechanizmem konstruktywnej indukcji cech dla wieloklasowych problemow
uczenia maszynowego, Ph.D. Thesis, Poznan University of Technology, 2000.

\bibitem{JelStefAI} Jelonek J., Stefanowski J.,
Feature subset selection for classification of
histological images. {\em Artificial Intelligence in Medicine}, {\bf 9}, 1997, 227-239.


\bibitem{JelStef97}  Jelonek, J., Stefanowski J., Using $n^2$-classifier to
solve multiclass learning problems. Technical Report, Poznan
University of Techonology, no RB-01/97,  November 1997.

%\bibitem{JelStefKr}  Krawiec K., Jelonek, J., Stefanowski J., Comparative
%study of feature subset selection techniques for machine learning
%tasks. In {\em Proceedings of VIIth Intelligent Information
%Systems} IIS'98 Malbork,15-19 June 1998, IPI PAN Press Warszawa
%1998, 68-77.


\bibitem{JelStef98}  Jelonek, J., Stefanowski J., Experiments on solving
multiclass learning problems by the n$^2$-classifier. {\em
Proceedings of 10th European Conference on Machine Learning},
Chemnitz,  Springer LNAI no. 1398, 1998, 172--177.

\bibitem{kohavi94}
John G., Kohavi R., Pfleger K., Irrelevant features and the subset
selection problem. {\em Proceedings of the Eleventh International
Machine Learning Conference}, New Brunswick NJ, Morgan Kaufmann,
1994, 121-129.



\bibitem{Kohavi95}  Kohavi R., Sommerfield D., Feature Subset Selection Using
the Wrapper Method: Overfitting and Dynamic Search Space Topology.
{\em Proceedings of the First International Conference on
Knowledge Discovery and Data Mining}, Montreal, AAAI Press, 1995,
192-197.


\bibitem{Stef01} Stefanowski J., Multiple and hybrid classifiers.
{\em Formal Methods and Intelligent Techniques
in Control, Decision Making, Multimedia and Robotics},
Post-Proceedings of 2nd Int. Conference, Warszawa, Polkowski L. (Ed.)
2001, 174--188.

\bibitem{Stef01hab}
Stefanowski J.,
{\em Algorithims of rule induction for knowledge discovery}.
(In Polish), Habilitation Thesis published as Series Rozprawy no. 361,
Poznan Univeristy of Technology Press, Poznan, 2001.

\bibitem{Weiss}
Weiss S.M., Kulikowski C.A.,
{\em Computer Systems That Learn: Classification and Prediction
Methods from Statistics, Neural Nets, Machine Learning and Expert
Systems}, Morgan Kaufmann, San Francisco, 1991.


\end{thebibliography}

\end{document}
