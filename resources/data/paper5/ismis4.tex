% This is a  paper on folder email categorization
% by J.Stefanowski and M.Zienkowicz
%
% This file follows the convention specified in
% the LaTeX macro package from Springer-Verlag
% for Lecture Notes in Computer Science,
% version 2.4 for LaTeX2e
%


\documentclass{llncs}


%\documentclass[runningheads]{llncs}


%In order to omit page numbers and running heads
%please use the following line instead of the first command line:
%\documentclass{llncs}.
%Furthermore change the line \pagestyle{headings} to
%\pagestyle{empty}.
%\input{psfig.sty}


\usepackage{url}


\begin{document}


%\pagestyle{headings}


\mainmatter


\title{Classification of Polish Email Messages: Experiments with Various
Data Representations}


\titlerunning{Classification of Polish Email Messages}


\author{Jerzy Stefanowski \and Marcin Zienkowicz}


\authorrunning{Stefanowski, Zienkowicz}


\institute{ Institute of Computing Science, Pozna\'{n} University of
Technology,\\ ul. Piotrowo 2, 60-965 Pozna\'{n}, Poland\\
\email{Jerzy.Stefanowski@cs.put.poznan.pl}, \email{  marcas@op.pl} }


\maketitle


\begin{abstract}
Machine classification of Polish language emails into user-specific folders
is considered. We experimentally evaluate the impact of different approaches
to construct data representation of emails on the accuracy of classifiers.
Our results show that language processing techniques have smaller influence
than an appropriate selection of features, in particular ones coming from
the email header or its attachments.
\end{abstract}



\section{Introduction}


An automatic categorization of emails into multiple folders could help users
in filtering too many incoming emails and organizing them in a structure
corresponding to different user's topics of interest. We are interested in
applying machine learning methods to find the user's folder assignment rules
based on the examples of his previous decisions \cite{linger,Manco}. Let us
remark the {\em folder categorization} is a more general problem than an
identification of spams only \cite{Bekkerman}. It is also different to
traditional text categorization as email messages are poorly structured
comparing to longer texts and are written in an informal way. Moreover,
there is no standard way of preparing email representation and there have
been no multiple folder benchmark data sets; a recent exception is the Enron
corpus~\cite{enron}.

Because of the page limit we reduce discussion on related works, see e.g.
reviews in \cite{Bekkerman,enron}. Shortly speaking, the current research
are focused on evaluating accuracy of various classifiers created by
learning approaches with indication to Na{\"i}ve Bayes, support vector
machines, k-nearest neighbor or boosted decision trees. Most of these
studies concerned spam filtering than folder categorization.

Unlike these research we focus our attention on issues of constructing
proper data representation, i.e. transforming emails into a data format
suitable for learning algorithms. The most commonly used is the {\em bag of
words} representation. Documents are represented as vectors of feature
values referring to  the importance of words selected from messages. The
features are mainly extracted from text parts of an email {\em subject} or
its {\em body}. Some researchers use also other features acquired from the
email header and from {\em attachment} files \cite{linger,enron,Manco}.
However, the role of these features in obtaining efficient classifiers is
not enough discussed.

Moreover, nearly all research concerned English emails. There are no such
works for the Polish language, which has more complex inflection and less
strict word order. Our previous study on Web page clustering showed that
preprocessing of documents in Polish influenced the results more than in
English~\cite{carrot}.

Therefore, the aim of our paper is to experimentally study the influence of
different ways of constructing data representation on the accuracy of
typical classifiers in the task of categorization Polish language emails
into user-specific folders. As there is no corpora of real email messages
written in Polish we have first to collect them. In data representations, we
want to explore the use of additional information available from the header
and attachment parts, which were not commonly applied in the previous
research. Moreover, we will examine the different ways of handling Polish
language stemming and word extraction.


\section{Data Sets and Their Representation}


We  collected three different Polish language email data sets. The first
one, further called {\em MarCas}, is a copy of personal emails received by
the second author of this paper since October 2003 till October 2004 and
contained 4667 messages. Nine different categories (folders) were used
within this period (the number of emails is reported in brackets): {\em
BooBoo} -- name of the student group (218 messages), {\em Humor} -- jokes,
stories (259), {\em pJUG} -- Java group (179), {\em Work} (1119), {\em
Family} (232), {\em Study} (587), {\em Friends} (289), {\em Spam} (1521),
{\em Other} (263). The other set, called {\em DWeiss}, is coming from our
close collaborator Dawid Weiss and contains the copy of his mailbox from a
longer period: October 2001-2004. Unfortunately, it concerns spam detection
only and contains 9725 messages from two categories legitimate {\em mail} --
5606 and {\em spam} -- 4119. Both collections contained complete information
about header sections and all attachments.

\newcommand{\ngrp}[1]{\textsf{\small#1}}
To extend information about multiple folders we constructed a third data set
as a collection of 36260 messages from newsgroups in Polish recorded since
March till April 2005. The {\em NewsGroups} contains messages from the
following groups: \ngrp{pl.comp.lang.java} (2971), \ngrp{pl.comp.lang.php}
(4896), \ngrp{pl.rec.kuchnia} -- Polish cuisine (4137),
\ngrp{pl.rec.muzyka.gitara} -- guitar music (7844),
\ngrp{pl.regionalne\-.po\-znan} -- regional news about the city of Poznan
(4896), \ngrp{pl.regionalne.poznan} -- news about Warsaw (6604),
\ngrp{pl.sci.fizyka} -- physics (2566), \ngrp{pl.sci.metamatyka} --
mathematics (2286). Unfortunately, available information on messages is
restricted, i.e.~no attachments were stored in the NewsGroups and header
parts were somehow restricted to main sections only, e.g.~a section on
message routing was missing.


Let us discuss a construction of data representations. The text content of
the {\em email body} part was the basis for extracting features for the bag
of words representation. The {\em header} part includes the set of pairs
({\em parameter-value}) -- their meaning is defined in RFC standards.
Similar to some researchers, we parsed the file to identify the following
elements: {\em Subject}, sender information ({\em From, Replay to}) and
recipient ({\em To, CC, Bcc}). The subject field was handled in the same way
as the body. Other elements were parsed to get the complete email addresses
or nicknames.  We also distinguished the case of multiple recipients, i.e.
each led to a separate feature. The recipient numbers were additional
numeric features. Unlike previous researchers we constructed next features
based on information about other parameters stored in the header. Quite
important for further analysis were {\em Received} parameters describing the
route of the message by computer servers. We also defined a numeric
attribute representing the number of servers, which were recorded for the
given message. Additional features included the message encoding, tools for
handling it, date, parts of the day, size of the message.


Besides these extended header features, we used more features about
attachments, i.e.~numeric ones representing the number of attached files of
a given type (zip, jpg, exe), names of the most frequent files. Contents of non-binary
attachment files were processed in the same way as the body or the subject and their terms
were used in the final representation.

While processing text parts we encountered the problems of {\em stemming},
{\em word extraction} and handling extra {\em formatting tags} in HTML.
Although previous researchers were not consistent on their role in English
emails \cite{enron,linger,Manco}, the proper use of stop words and stemming
is more influential on processing Polish documents \cite{carrot}. To examine
it in email classification we created different versions of each data set:
either with or without stemming. We used a quasi-stemmer for Polish called
{\em Stempelator} introduced by Weiss the text mining framework {\em
Carrot$^2$}; for more details see \url{www.carrot2.org}. Similarly we
studied two versions of word extraction: (1) with white characters
tokenization only, (2) also with handling the additional characters as
``@'', ``\$'' or ``!''. As to email bodies written in the HTML format we
also tested two techniques: either these tags were removed or not -- their
names were used as additional parameters. To sum up, depending on the above
choices  we had 8 versions of the data sets.

Because of page limits we skip details of all carried out experiments with
creating classifiers from these data sets and present conclusions only. For
all data sets tokenization should be extended by extra characters. Stemming
was particularly useful for DWeiss data while being not so significant for
others. HTML tags should be removed from MarCas and NewsGroups and should
stay in DWeiss data (perhaps because of their specific use in spam).

\setlength{\tabcolsep}{4pt}
\begin{table}[t]
\caption{Cardinality of features of given types in studied
data}\label{structure}
\begin{center}\scriptsize
\begin{tabular}{cccccc}
\hline\noalign{\smallskip} data  & header & servers & subject & body &
attach \\ \noalign{\smallskip} \hline \noalign{\smallskip} MarCas & 99 & 54
& 15 & 160 & 75 \\ DWeiss & 42 & 28 & 11 & 137 & 283 \\ NewsGroups & 24 & 0
& 12 & 198 & 0
\\ \hline
\end{tabular}
\end{center}
\end{table}
%\setlength{\tabcolsep}{1.4pt}


As the number of features in data representations was quite high (around few
thousands) we decided to reduce their number. Firstly, while creating bag of
words representation, we removed too frequent and too infrequent terms.
Then, we employed a typical feature selection mechanism based on the
following measures: {\em information gain}, {\em gain ratio} and $\chi^2$
statistics. The single features were ranked according to these measures. For
each data set, we selected the features occurring in the first positions
over mean values of the evaluation measures in these rankings.  The general
characteristics of reduced data is given in Table~\ref{structure}, where we
distinguished the following types of features depending on their origin:
body, subject, servers, attachments, header (additional features created
from other parts of the header, in particular senders).





\section{Experiments}


Three learning algorithms were chosen to construct email classifiers:
k-nearest neighbor (k-NN), decision tree  and Na{\"i}ve Bayes.  For running
our experiments we used the Weka toolkit\footnote{see
\url{www.cs.waikato.ac.nz/ml/weka}}. J48 (Weka C4.5 decision tree
implementation) was used with standard options. IBk -- a k-NN algorithm
implementation was tested with the number of neighbors equal to 1, 3 and 5
and weighted Euclidean distances. The best results were obtained for $k$ =
3. In Na{\"i}ve Bayes we used an option of discretizing numerical
attributes.


Performance of classifiers was evaluated with the following measures: {\em
total accuracy}, {\it recall}, {\it precision}  and {\em F-measure}.  In our
experiments we used {\em F-measure} with the aggregation function which
assigned equal importance to both  precision and recall. All these measures
were calculated for each category/ folder. Stratified 10-fold cross
validation was applied for all data sets and we report averaged results.
They are presented in Tables 2 and 3. Because of page limits, Table 3
contains results for 3-NN (IB3) only, as it performed similarly to J48 and
they both were better than Na{\"i}ve Bayes.




\begin{table}[t]
\setlength{\tabcolsep}{4pt} \caption{Total accuracy (\%) for different
versions of  data representations}\label{accuracy}
\begin{center}\scriptsize
\begin{tabular}{cccccc}
\hline\noalign{\smallskip} data  &  learning
&\multicolumn{3}{c}{feature set}\\
 set   & algorithm & subject &
body & all
\\ \noalign{\smallskip} \hline \noalign{\smallskip}
MarCas   & IB3 & 61.0 & 70.3 & 88.3 \\
  & J48 & 61.4 & 70.2 & 87.7 \\
  & Na{\"i}ve Bayes & 59.2 & 48.3 & 80.0 \\ \hline
 \noalign{\smallskip}
 DWeiss
   & IB3 & 82.9 & 83.5 & 98.6 \\
   & J48 & 82.7 & 82.6 & 98.4 \\
   & Na{\"i}ve Bayes & 82.3 & 79.3 & 82.2 \\ \hline
  \noalign{\smallskip}
 NewsGroups  & IB3 & 32.7 & 57.3 & 59.3 \\
   & J48 & 32.6 & 66.4 & 70.6 \\
   & Na{\"i}ve Bayes & 30.0 & 51.9 & 66.7 \\
 \hline
\end{tabular}
\end{center}
\end{table}






\setlength{\tabcolsep}{4pt}
\begin{table}[t]
\caption{Performance of classifiers for separate folders. For each measure,
three successive numbers [in \%] refer to three types of features sets:
subject, body and all features.}\label{RPF}
\begin{center}\scriptsize
\begin{tabular}{l l  r r r  r r r  r r r}
\hline \noalign{\smallskip}
      data set
        & folder
        & \multicolumn{3}{c}{Recall}
        & \multicolumn{3}{c}{Precision}
        & \multicolumn{3}{c}{F-measure}
\\ \noalign{\smallskip} \hline


\noalign{\smallskip} MarCas &
      BooBoo     & 84.4 & 70.9 & 86.9   & 97.2 & 97.2 & 97.2   & 88.9 & 69.9 & 87.9 \\
    & Humor      & 13.7 &  9.9 & 70.8   & 60.4 & 44.2 & 71.4   & 22.1 & 15.8 & 71.1 \\
    & pJUG       & 93.4 & 57.2 & 96.4   & 95.8 & 96.2 & 97.2   & 94.3 & 72.2 & 96.6 \\
    & Work           & 82.2 & 73.5 & 90.3   & 41.8 & 75.5 & 90.1   & 55.4 & 72.8 & 91.3 \\
    & Family     & 29.2 & 72.8 & 91.8   & 72.2 & 90.6 & 93.1   & 41.7 & 83.0 & 92.6 \\
    & Study      & 35.7 & 53.9 & 83.5   & 87.9 & 67.5 & 85.4   & 50.8 & 59.7 & 85.4 \\
    & Friends    & 11.3 & 30.4 & 49.5   & 80.0 & 58.7 & 53.7   & 19.5 & 40.1 & 49.6 \\
    & Spam       & 76.5 & 96.2 & 97.6   & 74.0 & 63.2 & 95.6   & 75.2 & 77.3 & 98.1 \\
    & Other      & 28.5 & 66.1 & 71.9   & 66.3 & 80.4 & 77.5   & 36.8 & 72.8 & 74.6 \\[2mm]
DWeiss
    & mail       & 85.1 & 79.3 & 98.4 & 81.3 & 98.3 & 98.7 & 86.0 & 83.1 & 98.8 \\
    & spam       & 76.3 & 80.2 & 97.8 & 82.4 & 72.2 & 97.9 & 77.6 & 83.2 & 98.4 \\[2mm]


NewsGroups
    & java       &  9.3 & 65.6 & 66.5 & 44.5 & 67.2 & 64.6 & 10.5 & 51.0 & 44.5 \\
    & php        & 38.6 & 78.7 & 80.0 & 47.0 & 78.3 & 78.5 & 38.6 & 63.3 & 64.4 \\
    & kuchnia    & 22.6 & 62.2 & 64.8 & 43.6 & 62.4 & 63.7 & 23.7 & 55.0 & 55.3 \\
    & gitara     & 40.4 & 70.7 & 76.1 & 28.4 & 65.7 & 75.3 & 40.3 & 62.3 & 65.8 \\
    & poznan     & 31.1 & 68.3 & 72.6 & 43.8 & 70.8 & 73.8 & 31.2 & 60.4 & 64.4 \\
    & warszawa   & 29.4 & 60.8 & 66.0 & 27.8 & 59.3 & 64.8 & 30.1 & 50.8 & 55.7 \\
    & fizyka     & 31.6 & 57.0 & 63.2 & 51.4 & 61.1 & 67.5 & 31.8 & 54.0 & 54.7 \\
    & matematyka &  9.1 & 61.6 & 68.4 & 23.4 & 65.0 & 71.4 &  9.4 & 56.6 & 58.1 \\ \hline
\end{tabular}
\end{center}
\end{table}

\section{Discussion of Results and Conclusions}

Let us discuss results of experiments. Comparing learning algorithms we
noticed that Na{\"i}ve Bayes produced significantly worse classification
accuracy than other algorithms for all versions of MarCas and DWeiss data
sets -- the accuracy was at least 10\% lower. For NewsGroups the difference
of accuracy was not so large. Looking for the best classifiers, we could say
that for e-mail data MarCas and DWeiss the performance of decision tree and
kNN algorithms was comparable, while J48 tree algorithm was slightly better
for NewsGroups. The problem of detecting spam was easier task than multiple
folder categorization.

More noticeable is the choice of features while creating the data
representation.  In Table~\ref{structure} we see the high number of finally
selected features about attachments (in particular for spam data), route of
emails by servers and other parts of header (e.g.~senders). Such information
was not available for Newsgroups, which may partly explain its lower
classification results. For all data sets we observed an increase of
classification accuracy when applying all these additional features. Using
exclusively either the subject or the body features led to lower accuracy
for all types of learned classifiers. For MarCas and NewsGroups data the
subject features were the least useful while for For DWeiss data they were
more important (perhaps it is connected with spam properties).


These observations are confirmed by the analysis of classifiers performance
in separate classes -- see Table 3.  It is clearly illustrated by changes of
the F-measure. Although these are results  for 3-NN classifier, quite
similar numbers were obtained for J4.8 decision trees, while slightly worse
again for Na{\"i}ve Bayes.  Considering other measures  it seems that
additional features affected recall more than precision. For some folders in
MarCas data the influence of using all features is really high, e.g.~see
recall in categories Humor, pJug, Work, Study or Family. On the other, it
had lower influence on precision, e.g.~in folder BooBoo, pJuG or Study. For
NewsGroups, the use of the subject features only did not contribute too much
to recall and precision, while the body features had also more influence on
recall than on precision. Depending on the given folder, some results for
the body features were quite comparable to the use of all features. For
detecting spams, i.e.~DWeiss data set, the use of all features increased
both precision and recall, and its influence was stronger for the spam
category than regular emails.  For precision the subject's features had
higher influence on detecting spam while the body features were quite
informative for classifying legitimate mails.


Comparing our results to related works on folder categorization of English
emails, we could notice that researchers stress mainly the role of the body
parts, see e.g.~\cite{enron}. Although some others say that features coming
from recipient or sender email parts are also useful (e.g. \cite{Manco}),
there is still lack of  experimental evaluations. An exception is the recent
study~\cite{enron}, the influence of different feature subsets on SVM
classifiers was studied with conclusion that combining additional features
improves the $F$ measure and the body or sender features are more suitable
than ones coming from the subject part.


Finally, we have  to admit that we used only 3 data sets, so we should be
cautious with making general conclusions. On the other hand, according to
our best knowledge this research problem has not be examined yet in the
context of Polish language and there have been no ready data sets. So, our
original contribution is collecting at least these three data sets having
different characteristics and making experiments with evaluating usefulness
of various ways of constructing data representations. Saying this we risk
conclusions that language processing techniques, as e.g. stemming, had
smaller influence on the final classifiers than extending feature subsets by
ones coming from the header part of an email  and information about its
attachments.



\medskip\noindent{\bf Acknowledgment}:
The research was supported from grant no. 3 T11C 050 26.



\begin{thebibliography}{10}




\bibitem{Bekkerman} R.~Bekkerman, A.~McCallum, G.~Huang: Automatic
categorization of email into folders: Benchmark Experiments on Enron and SRI
Corpora. U. Mass CIIR Report IR-418, 2004.


\bibitem{linger} J.~Clark, I.~Koprinska, J.~Poon: Linger -- a smart personal
assistant for e-mail classification. Proc. of the 13th Intern.
Conf. on Artificial Neural Networks (ICANN'03), 2003, 274--277.



\bibitem{enron} B.~Klimt, Y.~Yang: The enron corpus: a new dataset for email
classification research. In Proc. of the ECML'04 Conference, 2004, 217--226.

\bibitem{Manco} G. Manco, E. Masciari, M. Ruffolo, A. Tagarelli:  Towards an Adaptive Mail Classifier.
Workshop  "Apprendimento Automatico: Metodi ed Applicazioni",
 (AIIA �02). Siena,  September 10-13, 2002


%
%
\bibitem{carrot} J.~Stefanowski, D.~Weiss: Carrot$^2$
and language properties in Web Search Results clustering. In Proc.
of the 1st Atlantic Web Intelligence Conference AWIC-2003, LNCS
2663, 2003, 401-407.


%
%
\end{thebibliography}

\end{document}
